\chapter{Flight Test}
\label{flight-test}
The goal of flight testing was to validate the performance of the final system design. The drag polar estimation was chosen as the validation case, and was approached from each of the three coefficients. To this end, the final system was flown on multiple vehicles which each had a different roll in the validation routine.

\section{$C_{D_0}$ Validation}
Validation of the parasite drag coefficient was completed using a Finwing Universal Penguin FPV R/C aircraft\cite{penguinRC}. 

%todo: add picture of penguin plane
This model was selected because it has a large internal payload bay which has plenty of room for the system and had enough excess power to overcome additional drag. The validation method involved flying the base model and measuring the drag polar. Then, parasite drag was added in the form of streamers, similar to those used in amateur rocketry recovery. The benefit of this technique is that it does not modify either of the other two drag polar coefficients. A power law fit to the expected drag coefficient of the streamers was given in \cite{Auman2001}. The validation routine was flying the vehicle with and without streamers and seeing the horizontal shift in the drag polar.

\section{$K_1$ Validation}
The $K_1$ term of the drag polar is mainly driven by the $C_L$ for minimum drag. To validate this coefficient, a custom test vehicle was manufactured with two different wings of equal area. One wing had a NACA 63-014 airfoil cross section, which is a symmetric airfoil, meaning the vehicles $K_1$ value should be zero. The next wing had a NACA 63-614 airfoil, which has a design $C_L$ for minimum drag of 0.6. The validation method was flying both of these wings on the same vehicle, and seeing the vertical shift in the drag polar.
\section{$K_2$ Validation}