\chapter{Results}
\label{results}
The system was built and tested to prove functionality. After functionality testing was complete, the system was integrated into a 0.40-size Piper Cub R/C aircraft.

\begin{figure}[H]
\label{sysIntPics}
\begin{center}
\begin{minipage}[b]{0.45\linewidth}
  \centering
    \includegraphics[width=0.9\textwidth]{figures/sysInt1.jpg}
\end{minipage}
\begin{minipage}[b]{0.45\linewidth}
  \centering
    \includegraphics[width=0.9\textwidth]{figures/sysInt2.jpg}
\end{minipage}
\end{center}
\caption{System Integration into 0.40-size R/C Piper Cub}
\end{figure}
After integration, an initial test flight was conducted. Unfortunately, an electrical short-circuit caused the vehicle to  crash and corrupted some data, so no aerodynamic force estimation has been conducted. However, the micro-SD card did survive the crash, and provided enough data to show the system was collecting data as expected before the short-circuit.
\begin{figure}[H]
\label{sysIntPics}
\begin{center}
\begin{minipage}[b]{0.4\linewidth}
\label{preFlightFig}
  \centering
    \includegraphics[width=0.9\textwidth]{figures/preFlight.jpg}
    \caption{Pre-Flight System Checks}
\end{minipage}
\begin{minipage}[b]{0.4\linewidth}
  \centering
    \includegraphics[width=0.9\textwidth]{figures/crash.jpg}
    \caption{Crashed Vehicle in Water}
\end{minipage}
\end{center}
\end{figure}

 The small amount of data collected was analyzed using a user interface developed to rapidly process the data acquisition system's files. This user interface will allow designers to analyze data immediately following a test flight and make corrections to the test flight program while still at the test flight location.