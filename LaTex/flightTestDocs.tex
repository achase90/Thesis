\chapter{Flight Test Procedure}
This section documents the flight test procedure for using the data acquisition system. It is split into three time periods
\begin{enumerate}
\item Pre-Flight Preparation
\item Flying Field Procedure
\item Post-Flight 
\end{enumerate}
The testing procedure is split into these time categories to ensure the testing is efficient and that any unforseen circumstances can be dealt with quickly. Specifically , this guide applies to the Cal Poly Flight Lab testing a vehicle at the Cal Poly Educational Flight Range. In general, it is better to flight test early in the morning, since there will be less people at the field, winds will be calmer, and the sun won't be as harsh in the pilot's eyes.
\section{Pre-Flight Preparation}
The preparation work required for a test flight is often overlooked, and this section will documents how to effectively flight test a vehicle. The main purpose of pre-flight preparation is to minimize the possibility of problems that might force a test to be canceled after  already going to the field.
\subsection{Day Before Test}
The following should be done the day before a flight test:
\begin{enumerate}
\item Charge all flight battery packs, including any receiver or auxillary packs.
\item Charge the transmitter battery.
\item Ensure the airframe is structurally sound (wing tip test minimum.)
\item Verify radio system communicates properly, and the correct fail-safe is in place. \textbf{Important}: If the receiver has been used by Design/Build/Fly, the fail-safe should be changed to normal mode.
\item Verify control surface deflections matches desired directions, and all radio mixes work.
\item Verify the motor/propeller spin in the correct direction.
\item Pack a flight box, containing any necessary tools (recommend: screw drivers, various tapes, razor blades, CA glue, spare propeller, paper and pencil, as a minimum.)
\item Check the weather. The closest monitoring station to the field is \href{http://www.wunderground.com/cgi-bin/findweather/getForecast?query=35.326\%2C-120.738\&sp=KCASANLU17}{KCASANLU17}.
\item Check the SLO Flyers' flight schedule. Some days are reserved for certain events (glider competitions, etc.) and these need to be worked around.
\item Make sure all required personnel know when and where to meet, and there are sufficient rides to get to the field.
\item Create flight documentation that clearly lays out the test goals and how they will be accomplished. Print copies for all personnel.
\end{enumerate}
\subsection{Day Of Test}
The following should be done the morning of a flight test:
\begin{enumerate}
\item Pack flight batteries into flight box, including receiver and auxillary packs.
\item Pack battery charging equipment, with adapters and leads, if necessary.
\item Pack transmitter into box.
\item Double check control surface deflections and radio link.
\item Do a full system check, potentially including a short taxi test in the quad.
\item Do a final check that all equipment made it into vehicles, before leaving the lab.
\end{enumerate}

\section{Flying Field Procedure}
With the pre-flight preparation completed, the testing at the flying field should be fairly event free. Any problems that occur should be either fixable with the minimum supplies in the flight box, or the test should be canceled to minimize risk, and repairs done at the lab. Specific procedures at the flight field will depend on the test being conducted, but below are steps that apply to nearly all tests before flight.
\begin{enumerate}
\item Verify structural integrity using a wing tip test.
\item Verify motor/propeller are spinning in the correct direction.
\item Verify radio link and fail safe mode.
\item Verify control surface deflections match desired directions.
\item Verify center of gravity is at an appropriate location.
\item Create flight timer on the transmitter so the pilot knows how long the aircraft has been flying.
\end{enumerate}