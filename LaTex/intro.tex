%todo:i think this is fine
\section{Introduction}
\label{intro}
Aircraft designers often use model fits and ``rules of thumb'' to successfully complete designs, with many of these guidelines available in various design textbooks. \cite{raymer}$^,$\cite{nicolai2010fundamentals}$^,$\cite{roskam1985airplane} These models and practices have been established based on years of data analysis and validation that the designs perform as expected. However, the scope of these empirical design techniques is limited to the input of the regression model: an intelligent designer will not use a general aviation weight model for a transport category aircraft. To this end, there is a significant lack of small UAV-class guidelines for designing an aircraft.

 Accurate estimations of a small-scale vehicle's lift and drag characteristics are extremely critical to the aircraft designer, and affect both point performance (turn rates, climb rates, stall speeds, etc.) and mission performance (range and endurance). Much of the prediction tools available in the classic lift and drag textbooks from Hoerner\cite{hoernerDrag}$^,$\cite{hoernerLift} only apply to larger scale structures. In addition, drag prediction is extremely difficult on small vehicles, as some sources of drag are not easily modeled. For instance, actuator control horns and protruding screw heads are not typically included in the CFD \nomenclature{CFD}{Computational fluid dynamics} analysis of aircraft. However, for small vehicles, these sources of ``crud'' drag can be a significant portion of the total vehicle's drag value. 
 
 One advantage of small UAVs is that they are fairly inexpensive, which makes building multiple fully-functioning prototypes a viable option. The authors chose to take advantage of this fact and develop a flight data acquisition system with a primary goal of measuring the lift and drag characteristics of a small UAV. This would enable vehicle designers to build a conceptual-design-level prototype, and both develop and validate predictive, regression-based models. The system would also allow designers to conduct quantitative trade studies, such as the trade between drag reduction technologies (wheel pants, retractable landing gear, winglets, etc.) and the weight associated with them.\\

The authors also desired additional sensors to aid designers with more than just lift and drag characteristics. Some areas of interest are estimating stability and control derivatives, as well as possible control algorithm testing, in-flight thrust measurement, and payload integration capabilities. To accomplish these goals, sensors not directly necessary to lift and drag estimation were included in the overall system. The system was also designed in a manner that made it reconfigurable. This allows future designers to decide what combination of accuracy, risk, and sensing capabilities they need on a given flight test, and to then package the sensors in a manner that meets vehicle integration requirements.\\

For this paper, the authors chose to integrate the necessary sensors to estimate lift and drag forces, as well as those necessary for a basic INS\nomenclature{INS}{Inertial navigation system}, ambient temperature measurement, and servo and motor signals. The system will be validated by measuring the as-built drag polar of an R/C aircraft, as well as all other available states the system is capable of measuring.