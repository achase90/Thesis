\chapter{Introduction and Motivation}
\label{intro}
An accurate drag prediction is critical for conceptual aircraft design, aircraft mission planning, and predicting performance trends of comparable aircraft. To this end, industry spends an extensive amount of time and money developing wind tunnel models and executing wind tunnel tests. 
%todo:cite money or time in wind tunnel
Additionally, it is difficult to impossible to exactly scale down a vehicle, especially when features such as rivets, servo control horns, antennas, and air data probes are included. These differences between the model and the as-built aircraft can cause accuracy of the wind tunnel test to suffer. This inaccuracy 
%todo:cite scaling laws
inevitably leads to aerodynamic flight tests that attempt to quantify the as-built drag and lift characteristics of the vehicle.
\\
The flight test of full scale aircraft for drag polar prediction is generally conducted about a trimmed condition. That is, the aircraft is flown to an operating conditions dictated by the test plan, and sets the control surfaces such that there are no accelerations and no moments. Data is then collected for a set amount of time, without changing the operating condition. After the data is collected, the operating point is changed, the aircraft is trimmed at this new flight condition, and data is again collected. This process is repeated at various points in the aircraft's flight envelope until enough data is collected to estimate a drag polar.\\
Unfortunately, for many R/C aircraft and small UAVs, this procedure isn't feasible. First, these aircraft typically operate close to ground level, meaning there could potentially be both unsteady and turbulent winds, and a steady wind. R/C aircraft and small UAVs typically have much lower moments of inertias and mass than their full-scale counterparts, which means they will be affected much more by atmospheric disturbances than full-scale vehicles. Second, many of these aircraft have a line-of-sight communication link, and R/C aircraft in particular are flown in small patterns at a flight field. Even in the case of a steady atmosphere, R/C aircraft usually are not well trimmed, because by the time the pilot can see if the vehicle is trimmed, he has to turn around in the pattern. This thesis attempts to fix these problems, by allowing the pilot to fly in a more generic flight path, and not relying on a still atmosphere assumption.

\chapter{Method}

\label{background-information}
Some basic assumptions will apply throughout the modeling of dynamics in thesis. They are as follows:
\begin{enumerate}
\item The vehicle is a fixed mass.
\item Coriolis effects are negligible.
\item Thrust will be assumed to be 0.
\end{enumerate}
Note that a stationary atmosphere is not assumed. The fixed mass assumption is consistent with the electric aircraft used to test the system. At the altitude and speed at which the vehicles will be tested, Coriolis effects can be ignored\cite{klein2006aircraft}. The zero-thrust assumption is in place to minimize drag error. Any error in a measured state will decrease the accuracy of the drag measurement, and the accuracy of the drag measurement can be no better than the worst state measurement error. In-flight thrust is difficult to measure accurately, so a folding propeller will be used, and the motor will be turned off during data acquisition. This will allow the propeller to fold back, eliminating most of the wind-mill drag associated with a stalled propeller. 

\section{Reference Frames}
For this thesis, the reference frames used will follow those described in \cite{klein2006aircraft}, and will be repeated here for clarity.

\subsection*{North-East-Down (NED) Axes ($x_{ned}$, $y_{ned}$, $z_{ned}$)}
The NED axis system defines a local tangent plane on the Earth's surface, with the origin coinciding with the vehicle's center of gravity. The $\hat{i}$ vector points due north, the $\hat{j}$ vector points due east, and the $\hat{k}$ vector points towards the center of the earth, in accordance with the right-hand rule. This coordinate system is vehicle carried, meaning the origin is fixed, but the axis directions are independent of vehicle orientation.

\subsection*{Body Axes ($x_b$, $y_b$, $z_b$)}
The body axis system has its origin at the vehicle's center of gravity, with the $\hat{i}$ direction pointing out the vehicle's nose, the $\hat{j}$ direction pointing out the right wing, and the $\hat{k}$ direction pointing out the belly of the aircraft, in accordance with the right-hand rule. This coordinate frame is fixed to the body, meaning the aircraft's spatial orientation does not change the direction of the axes.

\subsection*{Stability Axes ($x_s$, $y_s$, $z_s$)}
The stability axes are defined with its origin coinciding with the center of gravity of the vehicle. This axis system has essentially the same directions as the body axes, except rotated about the body axis $\hat{j}$ through an initial angle-of-attack, $\alpha_{0}$. This inital angle-of-attack is defined at the beginning of a test maneuver and is then set for the remainder of the test, making it a body-fixed coordinate system. This system assumes no initial sideslip angle \cite{roskam2001airplane}.

\subsection*{Wind Axes ($x_w$, $y_w$, $z_w$)}
The wind axes are, again, a vehicle-carried coordinate system, meaning the origin of the wind axis also coincides with the center of gravity of the vehicle. However, the wind axes are not a body-fixed coordinate frame. The $\hat{i}$ direction points into the oncoming air, as seen from the vehicle. The $\hat{k}$ direction lies in the x-z plane of the body reference frame. The $\hat{j}$ direction is then defined to be out the right side of the vehicle, in order to follow the right hand rule. 

\section{Equations of Motion}
\label{sys-desc}
Newton's 2nd Law of Motion states
\begin{align}
\vec{F} &= \frac{d}{dt}(m\vec{V})
\end{align}
where $\vec{F}$ is the sum of all applied forces, $\vec{m}$ is the mass of the vehicle, and $\vec{V}$ is the vehicle's velocity. Using the fixed mass assumption, this reduces to 
\begin{align}
\vec{F} &= m\frac{d\vec{v}}{dt}\\
&= m\vec{a}
\end{align}

The applied forces on the vehicle are 
\begin{align}
\vec{F} &= \vec{F}_{A}+\vec{F}_{G}+\vec{F}_{T}
\end{align}

where $\vec{F_{A}}$ accounts for all aerodynamic forces acting on the vehicle, $\vec{F_{G}}$ is the force due to gravity, and $\vec{F_{T}}$ accounts for forces from the propulsion system.\\
Aerodynamic forces are described in the stability reference frame. In general, they are defined as

\begin{align}
\vec{F}_{A_S} &= D \hat{i}_s+Y \hat{j}_s+L \hat{k}_s
\end{align}

where $D$ is drag force, $Y$ is side force, and $L$ is lift force. \\
The gravitational force on the vehicle acts in the $+z_{ned}$ direction and is equal in magnitude to the vehicle's weight $W$, leading to
\begin{align}
\vec{F}_{G_{ned}} &= 0\hat{i}_{ned}+0\hat{j}_{ned}+W\hat{k}_{ned}
\end{align}
\\
In general, propulsive forces are modeled as
\begin{align}
\vec{F}_{T_b} &= T_x \hat{i}_b+T_y \hat{j}_b +T_z \hat{k}_b
\end{align}
where $T_x$, $T_y$, and $T_z$ are components of thrust in their respective body axis directions. However, as previously mentioned, propulsive forces are assumed to be $\vec{0}$ for this thesis.\\

The forces are combined and transformed into the body axes reference frame so that they align with the output of body mounted accelerometers. The combined equations of motion are then

\begin{align}
\vec{F}_{AERO_W} = DCM_{bw}^{-1}(m\vec{a} - DCM_{ib}\vec{F}_{GRAVITY_i})
\end{align}


\section{Kalman Filter Usage}
\label{kalman-filter}
This thesis utilizes multiple Kalman Filters to estimate both regression coefficients and improved states. The Kalman filter operates recursively on a time-series and provides an optimal estimate of the system state.

\subsection{Linear Kalman Filter}
A linear Kalman filter can be applied where the system in question can be described in the form \cite{welch1995introduction}

\begin{align}
x_k &= Ax_{k-1} + Bu_{k-1}+w_{k-1}
\end{align}
where $A$ is the state transition matrix, $x_{k-1}$ is the previous state, $B$ is the input matrix, $u_{k-1}$ is the input vector, and $w_{k-1}$ is random process noise.

The measured state is then 
\begin{align}
z_k &= Hx_k+v_k
\end{align} 

where $H$ is the output matrix and $v_k$ is measurement noise.

\indent
The Kalman filter operates in a predictor-corrector manner, where the predictor step is often called the \textit{a priori} estimate, and the corrector step is often called the \textit{a posteriori} estimate. The \textit{a priori} state estimate is calculated using prior states and inputs, while assuming no process noise

\begin{align}
\hat{x}^-_k &= A\hat{x}_{k-1}+Bu_{k-1}
\end{align}



The \textit{a priori} estimate of the covariance matrix is projected in a similar manner

\begin{align}
P^-_k &= AP_{k-1}A^T+Q
\end{align}

where Q the process noise covariance matrix.

The Kalman gain is calculated by combining the predicted, \textit{a priori} covariance matrix with the measurement noise covariance matrix $R$

\begin{align}
K_k &=P^-_kH^T(HP^-_kH^T + R)^{-1}
\end{align}

This optimal Kalman gain is then used to estimate the \textit{a posteriori} estimate of the state and covariance matrix

\begin{align}
\label{kalmanStateUpdate}
\hat{x}_k &=\hat{x}^-_k+K_k(z_k-y_k)
\end{align}

\begin{align}
P_k &= (I-K_kH)P^-_k
\end{align}

where $y_k$ is the predicted value of $z_k$ found using the output matrix $H$ and the \textit{a priori} state estimate
\begin{align}
y_k &= H\hat{x}^-_k
\end{align}
Note that Equation \ref{kalmanStateUpdate} is essentially a weighted average of a measured state and an expected state. The weighting is the Kalman gain, which is related to the ratio of confidence in the measured state and the expected state. For a 1-D case with equal confidence between the measured state and the expected state, the Kalman gain $K_k = 0.5$, and the Kalman filter becomes a simple and straight-forward average.


\subsection{Extended Kalman Filter}
\label{EKFTheory}
The Extended Kalman filter is used for a non-linear system and is essentially a linearization of a nonlinear plant. A non-linear system can be described as \cite{welch1995introduction}

\begin{align}
x_k &= f(x_{k-1},u_{k-1},w_{k-1})\\
z_k &= h(x_k,v_k)
\end{align}

The process noise $w_{k-1}$ and measurement noise $v_k$ are not known (or the Kalman filter would not be necessary), so the states are approximated assuming both noise sources are 0

\begin{align}
\tilde{x}_k &= f(\hat{x}_{k-1},u_{k-1},0)\\
\tilde{z}_k &= h(\tilde{x}_k,0)
\end{align}

The actual states are related to the approximate states by

\begin{align}
x_k &\approx\tilde{x}_k+A(x_k-\hat{x}_{k-1})+Ww_{k-1}\\
z_k &\approx\tilde{z}_k+H(x_k-\tilde{x}_{k-1})+Vv_k
\end{align}

where  the matrices $A$, $W$, $H$, and $V$ represent the different Jacobians matrices:
\begin{align}
A &= \frac{\partial f_i}{\partial x_j}(\hat{x}_{k-1},u_{k-1},0)\\
W &= \frac{\partial f_i}{\partial w_j}(\hat{x}_{k-1},u_{k-1},0)\\
H &= \frac{\partial h_i}{\partial x_j}(\hat{x}_{k},0)\\
V &= \frac{\partial h_i}{\partial v_j}(\hat{x}_{k},0)
\end{align}

The Extended Kalman Filter uses these linearized equations to perform the same process as the linear Kalman Filter. Again, the first step is to calculate the \textit{a priori} estimate of the state and the covariance matrix
\begin{align}
\hat{x}^-_k &=f(\hat{x}_{k-1},u_{k-1},0)\\
P^-_k  &= A_kP_{k-1}A^T_{k-1}+W_kQ_{k-1}W^T_k
\end{align}

Next, the Kalman gain is calculated
\begin{align}
K_k &=P^-_kH^T_k(H_kP^-_kH^T_k+V_kR_kV^T_k)^{-1}
\end{align}

The Kalman gain is then used to calculate the \textit{a posteriori} estimate of the state and covariance matrix

\begin{align}
\hat{x}_k &=\hat{x}^-_{k}+K_k(z_k-y_k)\\
\label{kalmanVariance}
P_k &=(I-K_kH_k)P^-_k
\end{align}

where $y_k$ is, as in the linear case, the predicted value of $z_k$, but calculated using the nonlinear output function and the \textit{a priori} state estimate

\begin{align}
y_k &= h(\hat{x}^-_k,0)
\end{align}

Unlike the linear Kalman filter, the Extended Kalman filter is not proven to be optimal. However, it has been utilized for a wide range of applications with excellent results.
