\chapter{Introduction and Motivation}
\label{intro}
An accurate drag prediction is critical for conceptual aircraft design, aircraft mission planning, and predicting performance trends of comparable aircraft. To this end, industry spends an extensive amount of time and money developing wind tunnel models and executing wind tunnel tests. 
%todo:cite money or time in wind tunnel
Additionally, it is difficult to impossible to exactly scale down a vehicle, especially when features such as rivets, servo control horns, antennas, and air data probes are included. These differences between the model and the as-built aircraft can cause accuracy of the wind tunnel test to suffer. This inaccuracy 
%todo:cite scaling laws
inevitably leads to aerodynamic flight tests that attempt to quantify the as-built drag and lift characteristics of the vehicle.
\\
The flight test of full scale aircraft for drag polar prediction is generally conducted about a trimmed condition. That is, the aircraft is flown to an operating condition dictated by the test plan, and sets the control surfaces such that there are no accelerations and no moments. Data is then collected for a set amount of time, without changing the operating condition. After the data is collected, the operating point is changed, the aircraft is trimmed at this new flight condition, and data is again collected. This process is repeated at various points in the aircraft's flight envelope until enough data is collected to estimate a drag polar.\\
Unfortunately, for many R/C aircraft and small UAVs, this procedure isn't feasible. First, these aircraft typically operate close to ground level, meaning there could potentially be both unsteady and turbulent winds, and a steady wind. R/C aircraft and small UAVs typically have much lower moments of inertias and mass than their full-scale counterparts, which means they will be affected much more by atmospheric disturbances than full-scale vehicles. Second, many of these aircraft have a line-of-sight communication link, and R/C aircraft in particular are flown in small patterns at a flight field. Even in the case of a steady atmosphere, R/C aircraft usually are not well trimmed, because by the time the pilot can see if the vehicle is trimmed, he has to turn around in the pattern. This thesis attempts to fix these problems, by allowing the pilot to fly in a more generic flight path, and not relying on a still atmosphere assumption.