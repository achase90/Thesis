\chapter{Summary}
\label{summary}
A flight data computer capable of measuring a small UAVs aerodynamic forces was developed. The system utilizes an Arduino Due as the main flight computer, and integrates both sensors necessary for aerodynamic force calculation, and additional sensors that provide interesting information about the vehicle. The system was manufactured in a PCB form to keep reliability high, integrated into a 0.60-size electric Piper Cub, and validated for accuracy and repeatability. The flight tests showed a $C_{D_0}$ accuracy of 6\% of the trailing cone's estimated drag coefficient, and a zero lift angle of attack accuracy of less than 1\% error compared to that estimated by XFOIL. The lift curve slope estimated from combining flight data with XFOIL analysis was, on average, accurate to 17\%, which indicates proper lift distribution estimation. Future work could include a sensor fusion algorithm, which would be developed to combine inertial sensors with the air data system and other available sensors in a manner similar to other current research,\cite{wvINSAirData}$^,$\cite{gtUKF} thus giving full situational awareness to the UAS. This situational awareness could allow stability and control derivative estimation, which the aircraft designer could use to size tail and control surfaces. In-flight dynamic thrust estimation is also possible, and could be validate against propeller data available from the University of Illinois at Urbana-Champagne.\cite{brandt2011propeller} Of most interest to the other, future work could utilize this thesis to estimate the lift and drag impact of difference vehicle configurations to quantitatively make trade studies early in the conceptual design phase. This could include configuration level trades, or subsystem level (no landing gear, normal gear, gear with wheel pants, retractable gear,etc.) trade studies, which could dramatically increase the current small scale aerodynamic knowledge base at Cal Poly.