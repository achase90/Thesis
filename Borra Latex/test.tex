$\left[ \begin{array}{c} q_2 \\ 2\zeta\omega_n \left( S_R \left\lbrace \frac{\omega^2_n}{2\zeta\omega_n} \left[ S_M(x^o_c)-q_1 \right] \right\rbrace -q_2 \right) \end{array} \right]$


The body state variables can be derived from the flight-path state variables ae follows:



In the development that follows, we let $T$ denote the propulsion force along the $\hat{b}_{x}$ direc- tion and $\ell_{p}^{c},\ m_{p}^{\mathrm{c}}, n_{p}^{\mathrm{c}}$ propulsion moments for roll, pitch, and yaw, respectively. Note that propulsion forces in the $\hat{b}_{y}$ and $\hat{b}_{B}$ directions have been neglected. Similarly, we let $m$ denote the mass of the aircraft, $I_{xx},\ I_{yy},\ I_{\mathrm{z}\mathrm{z}}$ the moments of inertia, $I_{xE}$ the $xz$ product of inertia, and $g$ the gravitational acceleration. Note that products of inertia $I_{xy}$ and $I_{y\mathrm{z}}$ are zero, due to our assumption of aircraft symmetry about the $xz$ plane.

4.1.1.1 Atmospheric Motion

We model atmospheric motion relative to the earth with three translational velocities ( $v_{x}^{\mathrm{n}},\ v_{y\iota}^{a}v_{\mathrm{z}}^{a}$ for wind from the North, East and down, respectively) and three rotational velocities $(\omega_{x;}^{a}\omega_{y}^{a},\ \omega \mathcal{T}$ for wind about axes opposite to the unit vectors for North, Eastl and down, respectively). These components are rotated through the body Euler angles to obtain equivalent body components of atmospheric motion as follows:

$\left\{\begin{array}{l}
u^{a}\\
v^{a}\\
w^{a}
\end{array}\right\}=\left\{\begin{array}{lll}
\mathrm{c}\mathrm{o}\mathrm{s}\theta \mathrm{c}\mathrm{o}\mathrm{s}\psi & \mathrm{c}\mathrm{o}\mathrm{s}\theta s\mathrm{i}\mathrm{n}\psi & -\mathrm{s}\mathrm{i}\mathrm{n}\theta\\
-\mathrm{s}\mathrm{i}\mathrm{n}\psi \mathrm{c}\mathrm{o}\mathrm{s}\phi+\mathrm{s}\mathrm{i}\mathrm{n}\theta \mathrm{c}\mathrm{o}\mathrm{s}\psi \mathrm{s}\mathrm{i}\mathrm{n}\phi & \mathrm{c}\mathrm{o}\mathrm{s}\psi \mathrm{c}\mathrm{o}\mathrm{s}\phi+\mathrm{s}\mathrm{i}\mathrm{n}\theta \mathrm{s}\mathrm{i}\mathrm{n}\psi \mathrm{s}\mathrm{i}\mathrm{n}\phi & \mathrm{c}\mathrm{o}\mathrm{s}\theta \mathrm{s}\mathrm{i}\mathrm{n}\phi\\
\mathrm{s}\mathrm{i}\mathrm{n}\psi \mathrm{s}\mathrm{i}\mathrm{n}\phi+\mathrm{s}\mathrm{i}\mathrm{n}\theta \mathrm{c}\mathrm{o}\mathrm{s}\psi \mathrm{c}\mathrm{o}\mathrm{s}\phi & -\mathrm{c}\mathrm{o}\mathrm{s}\psi \mathrm{s}\mathrm{i}\mathrm{n}\phi+\mathrm{s}\mathrm{i}\mathrm{n}\theta \mathrm{s}\mathrm{i}\mathrm{n}\psi \mathrm{c}\mathrm{o}\mathrm{s}\phi & \mathrm{c}\mathrm{o}\mathrm{s}\theta \mathrm{c}\mathrm{o}\mathrm{s}\phi
\end{array}\right\}\left\{\begin{array}{l}
v_{x}^{a}\\
v_{y}^{a}\\
v_{\mathrm{z}}^{\mathrm{d}}
\end{array}\right\}$
\begin{flushright}
(4.16)
\end{flushright}\begin{center}
62
\end{center}

\end{document}
